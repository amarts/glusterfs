\documentclass{book}[12pt]
\usepackage{graphicx}
% \usepackage{fancyhdr}

% \pagestyle{fancy}
\begin{document}

% \headheight 117pt
% \rhead{\includegraphics{zr-logo.eps}}

\author{Gluster}
\title{GlusterFS 1.3 Hacker's Guide}
\date{June 1, 2007}

\maketitle
\frontmatter
\tableofcontents

\mainmatter
\chapter{Introduction}

\section{Coding guidelines}
GlusterFS uses Git for version control. To get the latest source do:
\begin{verbatim}
  $ git clone git://git.sv.gnu.org/gluster.git glusterfs
\end{verbatim}
\noindent
GlusterFS follows the GNU coding
standards\footnote{http://www.gnu.org/prep/standards\_toc.html} for the
most part.

\chapter{Major components}
\section{libglusterfs}
\texttt{libglusterfs} contains supporting code used by all the other components. 
The important files here are:

\texttt{dict.c}: This is an implementation of a serializable dictionary type. It is
used by the protocol code to send requests and replies. It is also used to pass options
to translators.

\texttt{logging.c}: This is a thread-safe logging library. The log messages go to a
file (default \texttt{/usr/local/var/log/glusterfs/*}).

\texttt{protocol.c}: This file implements the GlusterFS on-the-wire
protocol. The protocol itself is a simple ASCII protocol, designed to
be easy to parse and be human readable.

A sample GlusterFS protocol block looks like this:
\begin{verbatim}
  Block Start                            header
  0000000000000023                       callid
  00000001                               type
  00000016                               op
  xxxxxxxxxxxxxxxxxxxxxxxxxxxxxxxx       human-readable name
  00000000000000000000000000000ac3       block size
  <...>                                  block
  Block End
\end{verbatim}

\texttt{stack.h}: This file defines the \texttt{STACK\_WIND} and
\texttt{STACK\_UNWIND} macros which are used to implement the parallel
stack that is maintained for inter-xlator calls. See the \textsl{Taking control
of the stack} section below for more details.

\texttt{spec.y}: This contains the Yacc grammar for the GlusterFS
specification file, and the parsing code.


Draw diagrams of trees
Two rules:
(1) directory structure is same
(2) file can exist only on one node

\section{glusterfs-fuse}
\section{glusterfsd}
\section{transport}
\section{scheduler}
\section{xlator}

\chapter{xlators}
\section{Taking control of the stack}
One can think of STACK\_WIND/UNWIND as a very specific RPC mechanism.

% \includegraphics{stack.eps}

\section{Overview of xlators}

\flushleft{\LARGE\texttt{cluster/}}
\vskip 2ex
\flushleft{\Large\texttt{afr}}
\vskip 2ex
\flushleft{\Large\texttt{stripe}}
\vskip 2ex
\flushleft{\Large\texttt{unify}}

\vskip 4ex
\flushleft{\LARGE\texttt{debug/}}
\vskip 2ex
\flushleft{\Large\texttt{trace}}
\vskip 2ex
The trace xlator simply logs all fops and mops, and passes them through to its child.

\vskip 4ex
\flushleft{\LARGE\texttt{features/}}
\flushleft{\Large\texttt{posix-locks}}
\vskip 2ex
This xlator implements \textsc{posix} record locking semantics over
any kind of storage.

\vskip 4ex
\flushleft{\LARGE\texttt{performance/}}

\flushleft{\Large\texttt{io-threads}}
\vskip 2ex
\flushleft{\Large\texttt{read-ahead}}
\vskip 2ex
\flushleft{\Large\texttt{stat-prefetch}}
\vskip 2ex
\flushleft{\Large\texttt{write-behind}}
\vskip 2ex

\vskip 4ex
\flushleft{\LARGE\texttt{protocol/}}
\vskip 2ex

\flushleft{\Large\texttt{client}}
\vskip 2ex

\flushleft{\Large\texttt{server}}
\vskip 2ex

\vskip 4ex
\flushleft{\LARGE\texttt{storage/}}
\flushleft{\Large\texttt{posix}}
\vskip 2ex
The \texttt{posix} xlator is the one which actually makes calls to the
on-disk filesystem. Currently this is the only storage xlator available. However,
plans to develop other storage xlators, such as one for Amazon's S3 service, are
on the roadmap.

\chapter{Writing a simple xlator}
\noindent
In this section we're going to write a rot13 xlator. ``Rot13'' is a
simple substitution cipher which obscures a text by replacing each
letter with the letter thirteen places down the alphabet. So `a' (0)
would become `n' (12), `b' would be 'm', and so on.  Rot13 applied to
a piece of ciphertext yields the plaintext again, because rot13 is its
own inverse, since:

\[
x_c = x + 13\; (mod\; 26)
\]
\[
x_c + 13\; (mod\; 26) = x + 13 + 13\; (mod\; 26) = x
\]

First we include the requisite headers.

\begin{verbatim}
#include <ctype.h>
#include <sys/uio.h>

#include "glusterfs.h"
#include "xlator.h"
#include "logging.h"

/*
 * This is a rot13 ``encryption'' xlator. It rot13's data when 
 * writing to disk and rot13's it back when reading it. 
 * This xlator is meant as an example, not for production
 *  use ;) (hence no error-checking)
 */

\end{verbatim}

Then we write the rot13 function itself. For simplicity, we only transform lower case
letters. Any other byte is passed through as it is.

\begin{verbatim}
/* We only handle lower case letters for simplicity */
static void 
rot13 (char *buf, int len)
{
  int i;
  for (i = 0; i < len; i++) {
    if (isalpha (buf[i]))
      buf[i] = (buf[i] - 'a' + 13) % 26;
    else if (buf[i] <= 26)
      buf[i] = (buf[i] + 13) % 26 + 'a';
  }
}
\end{verbatim}

Next comes a utility function whose purpose will be clear after looking at the code
below.

\begin{verbatim}
static void
rot13_iovec (struct iovec *vector, int count)
{
  int i;
  for (i = 0; i < count; i++) {
    rot13 (vector[i].iov_base, vector[i].iov_len);
  }
}
\end{verbatim}

\begin{verbatim}
static int32_t
rot13_readv_cbk (call_frame_t *frame,
                 call_frame_t *prev_frame,
                 xlator_t *this,
                 int32_t op_ret,
                 int32_t op_errno,
                 struct iovec *vector,
                 int32_t count)
{
  rot13_iovec (vector, count);

  STACK_UNWIND (frame, op_ret, op_errno, vector, count);
  return 0;
}

static int32_t
rot13_readv (call_frame_t *frame,
             xlator_t *this,
             dict_t *ctx,
             size_t size,
             off_t offset)
{
  STACK_WIND (frame,
              rot13_readv_cbk,
              FIRST_CHILD (this),
              FIRST_CHILD (this)->fops->readv,
              ctx, size, offset);
  return 0;
}

static int32_t
rot13_writev_cbk (call_frame_t *frame,
                  call_frame_t *prev_frame,
                  xlator_t *this,
                  int32_t op_ret,
                  int32_t op_errno)
{
  STACK_UNWIND (frame, op_ret, op_errno);
  return 0;
}

static int32_t
rot13_writev (call_frame_t *frame,
              xlator_t *this,
              dict_t *ctx,
              struct iovec *vector,
              int32_t count, 
              off_t offset)
{
  rot13_iovec (vector, count);

  STACK_WIND (frame, 
              rot13_writev_cbk,
              FIRST_CHILD (this),
              FIRST_CHILD (this)->fops->writev,
              ctx, vector, count, offset);
  return 0;
}

\end{verbatim}

Every xlator must define two functions and two external symbols. The functions are 
\texttt{init} and \texttt{fini}, and the symbols are \texttt{fops} and \texttt{mops}.
The \texttt{init} function is called when the xlator is loaded by GlusterFS, and 
contains code for the xlator to initialize itself. Note that if an xlator is present
multiple times in the spec tree, the \texttt{init} function will be called each time
the xlator is loaded.

\begin{verbatim}
int32_t
init (xlator_t *this)
{
  if (!this->children) {
    gf_log ("rot13", GF_LOG_ERROR, 
            "FATAL: rot13 should have exactly one child");
    return -1;
  }

  gf_log ("rot13", GF_LOG_DEBUG, "rot13 xlator loaded");
  return 0;
}
\end{verbatim}

\begin{verbatim}

void 
fini (xlator_t *this)
{
  return;
}

struct xlator_fops fops = {
  .readv        = rot13_readv,
  .writev       = rot13_writev
};


\end{verbatim}

\end{document}

